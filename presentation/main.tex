\documentclass{beamer}

\usetheme[subsectionpage=progressbar]{metropolis}
\setbeamertemplate{section in toc}[sections numbered]
\setbeamertemplate{subsection in toc}[subsections numbered]

\setbeamertemplate{frame numbering}[fraction]
% \useoutertheme{miniframes}
% \useinnertheme{rounded}
% \usefonttheme{metropolis}
% \usecolortheme{orchid}
% \setbeamercolor{background canvas}{bg=white}
% \usecolortheme{wolverine}

\makeatletter
\setbeamertemplate{section page}{
  \centering
  \begin{minipage}{22em}
    \raggedright
    \usebeamercolor[fg]{section title}
    \usebeamerfont{section title}
    \thesection.~\insertsectionhead\\[-1ex]
    \usebeamertemplate*{progress bar in section page}
    \par
    \ifx\insertsubsectionhead\@empty\else%
      \usebeamercolor[fg]{subsection title}%
      \usebeamerfont{subsection title}%
      \thesection.\thesubsection~\insertsubsectionhead
      \fi
  \end{minipage}
  \par
  \vspace{\baselineskip}
}
\makeatother

% \hypersetup{pdfstartview={Fit}} % fits the presentation to the window when first displayed

\title{Longest Common Subsequences}
\subtitle{Seminar 2}
\author{Joris LIMONIER}
\institute{University of Luxembourg}
\date{March 31, 2021}



\begin{document}

\maketitle
% \begin{frame}[plain]
% \end{frame}

\begin{frame}{Table of Contents}
    \tableofcontents
\end{frame}


\section{Introduction}
\subsection{What are LCS ?}
\begin{frame}
    LCS = Longest Common Subsequence(s)
\end{frame}

\subsection{Does it even matter ?}
\begin{frame}
\end{frame}

\begin{frame}
\end{frame}

\begin{frame}
\end{frame}

\section{How to find LCS ?}
\subsection{Building the table}
\begin{frame}
\end{frame}

\subsection{Crawling back up the table}
\begin{frame}
\end{frame}

\section{Data analysis of LCS results}
\begin{frame}
\end{frame}

\begin{frame}[standout]
    Thank you
\end{frame}

\end{document}